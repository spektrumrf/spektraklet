\documentclass{spektraklet}

\begin{document}

%----------------------------------------------------------------------
%	arguments:
%		#1	publication number (eg. 1/2015, 2/2015, 3/2015 etc.),
%			the year is filled in programatically.
%
%		#2	the width of the cover image
%
%		#3	the path to the cover image
%
%		#4	optional argument with offsets for the cover image
%----------------------------------------------------------------------
\titlepage{1}{\paperwidth}{images/Parm.jpg}[0, -1cm]
%\titlepage{1}{0.75\linewidth}{images/dog.jpg}



%----------------------------------------------------------------------
%	There are a couple of fields in the content page that is changable,
%	and some of them are mandatory. The default values are defined in
%	the 'spektraklet.cls' file.
%
%	Mandator:
%		\ChiefEditor{name}
%		\ManagingEditor{name}
%		\Editors{name1\\name2\\name3}	the last name shouldn't be followed by a \\
%
%	Optional:							Default value:
%		\PublicationInformation[text]		Àr ett språkrör för de som studerar - eller låter bli
%											att studera - matematik, fysik, kemi eller datavetenskap
%											på svenska vid Helsingfors universitet
%		\PublicationSupport[text]			Spektraklet får HUS-stöd för föreningstidningar
%		\PublisherName[text]				Spektrum rf
%		\PublisherAddress[text]				Kemiska institutionens svenska avdelning
%		\PublisherPostalOfficeBox[text]		PB 55
%		\PublisherPostalCode[text]			00014 Helsingfors universitet
%----------------------------------------------------------------------


\ChiefEditor{Sandra Kulla}
\ManagingEditor{Jeremias Berg}
\Editors{
Sebastian Björkqvist\\
Jere Mannisto\\
Susanna Liesipohja\\
Kadri Kiilas\\
Robert Mäkitie\\
Annaleena Rajala
}
\CoverPageAuthor{Jeremias Berg}


% Include the content page.
\contentpage




\begin{ledaren}{Sandra}

\wrappicture{images/sandrachef.jpg}[3cm]

God jul på er alla!

Nu är det jul igen och nu är det jul igen och julen varar väl till påska, och så vidare. Tenterna är snart skrivna och många far hem till sina familjer för att tillbringa julen med dem. Julen är en tid för gemenskap och stillhet. Även i Spektrum firas det lite jul, i form av julfest, med mycket gemenskap och lite mindre stillhet.

Det att julen närmar sig betyder också att det nya året är på väg, vilket i sin tur betyder att en ny styrelse snart ska få ta över ansvaret inom Spektrum. Detta innebär självfallet att även jag ska stiga ner från min post som chefredaktör. Nästa år tar Celina över ansvaret för mig och får en ny härlig redaktion som hjälper henne.

I detta, mitt sista spektrakel, finns en stor del av de artiklar som under året publicerats, samt härligt svammel. En fantastisk ordförandespalt skriven av en nioårig Jonas, finns också att hittas här i. Med andra ord, massor av julläsning inför det långa jullovet.

Jag vill till slut tacka redaktionen för att ha hjälpt mig detta år och för de otroliga texter som ni skrivit. Så god jul och gott nytt år allesammans!

\end{ledaren}




\begin{ordforandespalten}{Jonas, 9 ÅR}

\wrappicture{images/jonasordf.jpg}[][R]

Snögubben Köld berättar
Hej, jag heter köld och jag är en snögubbe. Jag finns i Virkby på Westerlunds gård. Det var två barn som heter Jonas och Kati som gjorde mig. Jag är virkbys största snögubbe. Jag är 1m 50cm hög och jag har en hatt en kepp en kravatt och glasögon. En dag såg jag ett flygplan flyga över mig och då önskade jag att jag kunde flyga.

Plötsligt kom det en trollkarl och gav åt mig en önskning. Jag sa åt honom att skulle önska att jag kunde flyga. Plötsligt hade jag vingar och jag flög till nordpolen. Men plötsligt mindes jag att det var en dröm.

Jag är rädd bara för en sak och det är våren för då smälter jag. Jag önskar att nån skulle kasta vatten på mig att jag skulle bli is och inte smälta så fort. På julen fick jag min önskning för att Westerlunds familjen fick till julklapp ett ämbar kalt vatten och en lapp som det stod att: Kasta vattnet på snögubben. Familjen Westerlund kasta vattnet på mig så som det sto i lappen och nästa natt var jag redan is. Men för att is smälter också så när våren kom smalt jag men lite senare enn vanligt.

God jul till er alla!
	
\end{ordforandespalten}



\begin{artikel}{Mattips för fattiga studerande}{Susanna}
\begin{twocolumns}
Som studerande, som för första gången bor på egen hand kan det vara svårt att veta hur man kan äta bra mat, som dessutom är billig. På dagarna sparar man ju pengar genom att äta billigt vid Unicafé, men så att middagarna och veckoslutsmåltiderna inte enbart består av fryspizza och makaroner, så kommer här några allmänna tips för hur du kan köpa bra mat på ett sådant sätt att kostnaderna hålls nere.

\columnpicture{images/rea.jpg}
\textbf{Spana efter erbjudanden}

För att få veta om erbjudanden, så kommer man här lättast undan om man helt enkelt tar bort "Ingen reklam, tack"-skylten från postlådan. Alternativt, om man vill undvika pappersskräp, så kollar man upp erbjudanden på butikernas egna hemsidor. I alla fall Lidl brukar ha klart och tydligt sina erbjudanden på nätsidorna, och här gäller det att hålla koll på matrean som alltid är i kraft från torsdag till söndag, samt lördagsrean.
Det lönar sig även att hålla utkik efter röda/orangea realappar, dvs. mat med nedsatt pris på grund av att sista försäljningsdagen närmar sig. Det lönar sig att vara ute efter dessa tidigt på dagen, innan de blir uppköpta, fast i små butiker som har långa öppettider brukar man också ha bra chans att hitta dessa produkter sent på kvällen, strax innan stängningsdags.

\textbf{Handla sällan}

När man går till butiken så slinker det ofta med något extra som man inte hade planerat att köpa. Fastän man inte tycker att man köper nåt extra så verkar ändå summan för veckans matinköp bli lägre om man bara handlar en gång i veckan, jämfört med om man flera gånger i veckan går till butiken och bara köper lite varje gång. Möjligtvis är man mer medveten om hur mycket pengar det går åt när man handlar en större mängd mat åt gången, och på så sätt gallrar man bort som som inte är lika nödvändigt.

Hur som helst, billigast kommer man undan om man planerar hela veckans mat på en gång, handlar allt vad man behöver på en enda gång och undviker butiken så långt man bara kan resten av veckan.

\textbf{Köp säsongsbetonat}

När det gäller frukter och grönsaker så gäller det att satsa på sånt som för tillfälligt är i säsong, speciellt om man vill hålla sig till inhemska grönsaker. Det betyder att det på sommaren lönar sig att satsa på tomat, gurka och sallader, emedan det på hösten och vintern är kål och rotfrukter som gäller. Rivna morötter är ett enkelt och billigt grönsaksalternativ under det mörkare halvåret. Kål har en bra smak som rå, eller så kan den strimlas tunt och stekas.

\columnpicture{images/rodflugsvamp.jpg}

\textbf{Åk ut i skogen}

När det kommer till bär och svamp, så kommer man allra billigast undan om man själv tar sig ut i skogen med ett ämbar. Men om man lätt går vilse och har svårt att se skillnad på Karl Johan och flugsvamp, så är detta något som kanske inte rekommenderas. Bär är ännu ganska lätta att känna igen, men när det gäller svamp så borde man nästan ha med någon som vet vad de gör. Kanske någon snäll biolog vill hjälpa dig?
\columnpicture{images/basilika.jpg}
\textbf{Odla eget}

Örter är enkla att odla hemma i lägenheten, och den mer avancerade kan även försöka sig på tomat eller paprika! På IKEA säljs det örtodlings-kits för nybörjare, men redan en butiksköpt örtkruka kan hållas vid liv förvånansvärt länge. Jag blev till sist tvungen att slänga min basilikakruka (köpt från S-market) eftersom jag inte skulle vara hemma på några veckor och hade inga möjligheter att vattna den. Allt jag hade gjort för att hålla den vid liv var att sätta den lilla plastkrukan i en kopp och vattna den lite emellanåt.

\textbf{Välj dina proteinkällor}

Kött är dyrt, och här lönar det sig speciellt att hålla utkik efter erbjudanden. Har man frys så är man lyckligt lottad, då lönar det sig att ta vara på erbjudanden och köpa extra för att frysa ner. Kolla även på kilopriset innan du köper, inte priset per paket. Om man bortser från specialerbjudanden, så verkar de billigaste kötten vara marinerade kycklinglår, malet kött (speciellt gris-nöt) och diverse "sämre" kött, som lätt blir segt vid tillagning. När det gäller kött som lätt blir segt vid tillagning, såsom karelsk stek, så rekommenderar jag att man använder köttet till grytor. Efter fyra timmar på låg värme brukar även de segaste kötten bli härligt möra!

En bra och billig proteinkälla är ägg. Det är enkelt att fixa ihop en omelett till middag, och dessutom kan man slänga med nästan vad som helst som man hittar i kylskåpet. Använd fantasin och experimentera!

När det gäller mjölkprodukter, så är kvarg en bra proteinkälla. Naturell kvarg är billigt, och den kan enkelt smaksättas med bär, frukt eller saftkoncentrat.

För att denna artikel inte bara ska vara fylld med allmänna riktlinjer, så är här ett basrecept för maletköttbiffar, som jag även själv använder. Man kan enkelt variera receptet genom att sätta till lök, vitlök, auraost, etc. Ja, det är bara att experimentera! Men jag garanterar inte att resultatet av experimenten blir bra...


\textbf{Basrecept för maletköttbiffar}

400 g malet kött 

1 ägg 

1 tsk salt 

1/3 tsk svartpeppar 

smör eller olja till stekning 

Ta fram köttet i rumstemperatur en stund innan tillagning. Förvärm ugnen till 175 grader. Knåda ihop alla ingredienser förutom smör/olja tills du får en fast smet och forma 4-6 biffar. Bryn biffarna i smör eller olja så dom får lite färg (de får fortfarande vara röda inuti). Sätt biffarna i en ugnsform och in i ugnen i 15-20 minuter. Tillagningstiden varierar beroende på biffarnas tjocklek och hur länge man steker dem före. Är man osäker så kan man skära upp biffen och kolla att den inte är röd inuti.
Tips: En bra idé är att steka lök i det överblivna stekspadet från biffarna. Då får löken en riktigt god smak!


\end{twocolumns}
\end{artikel}


\begin{artikel}{Helt skit artikel, del 2: Serotonin - \\ En historia om diarré, LSD och antidepressanter}{Jere}[]

I förra delen i skitserien behandlades det eventuella sambandet mellan nationella mängden intagen koffein och förbruket av toalettpapper. Även denna artikel tangerar ämnet, men ur neurokemisk synvinkel. Människan är ett komplext neurobiologiskt system som regleras med flera olika signalämnen, såsom serotonin, dopamin och acetylkolin. De reglerar humör, aptit, sömn, inlärning och minne, samt flera fysiologiska funktioner. Ja, men VARFÖR bry sig? I moderna tider utsätts vi för en ökande mängd olika preparat, som med eller utan avsikt påverkar våra handlingar i vardagliga situationer. Därmed är det viktigt att veta HUR de påverkar oss. Denna artikel har serotonin i focus.

Historien börjar inte direkt med skit, utan med själva tarmen. Ca år 1950 upptäckte italienska forskande att tarmcellerna innehöll ett nytt okänt ämne, serotonin. Ungefär samtidigt hittade två andra forskargrupper, oberoende av varandra, serotonin i trombocyter, dvs blodplättar, och i hjärnan. Det var uppenbart att serotonin hade en viktig biologisk roll.[1]

Några år tidigare, 1943, hade schweizaren Albert Hofmann upptäckt de psykedeliska effekterna av LSD.2 Många forskade gjorde iakttagelsen att LSD och serotonin hade en mycket liknande struktur, bild 2. Om LSD hade en mycket stark effekt på sinnet redan med små doser, så kanske även serotonin hade en neurologisk funktion? Hypotesen var alltså att patienter med tarmsjukdomar skulle ha höjda eller sänkta serotoninhalter, och därmed mentala problem. Mycket riktigt bevisade man 1954 att patenter med diarré pga en viss typ av tarmcancer hade höjda halter av serotonin, men ingen koppling till mentala sjukdomar kunde finnas. På 1950- och 1960-talet behandlades dessa cancerpatenter med fenklonin, ett ämne som inhiberar biosyntesen av serotonin. Behandlingen lättade på diarrésymptomen men patienterna fick symptom som liknade depression. Detta var början för hypotesen att depression orsakas av serotoninbrist i hjärnan. Med andra ord var tanken att depression kunde botas genom att höja halten av serotonin i hjärnan.


\picture{images/serotonin.png}[0.5\columnwidth]

Men hur borde man gå till väga för att öka halten serotonin? Lösningen till dilemmat kräver kunskap om serotoninets roll i centrala nervsystemet, bild 3. Nervceller är inte direkt kopplade till varandra, utan i stället förmedlar de signaler till varandra över små klyftor, s.k. synapser. Olika signaler förmedlas i synapserna med flera olika signalmolekyler, till vilka även serotonin hör. Nervcellerna lagrar serotonin i s.k. vesiklar, som är speciella membraner (”bubblor”). Vesiklarna transporteras till ytan när en nervcell skall förmedla en signal, vilket resulterar i en utlösning av serotonin. Serotonin binder sig till receptorer i mottagande cellen. Det finns sju kända receptortyper för serotonin, och dessa kan ha flera underklasser. Ju mer serotonin som binder sig till receptorerna, desto starkare är signalen som mottagande cellen förmedlar. Till slut lossnar serotonin från receptorerna och upptas tillbaka till ursprungliga nervcellen med ett transportprotein, som fungerar som en pump. Upptaget serotonin bryts ner i nervcellen. Upptagningsmekanismen ser till att signalen inte förmedlas en längre tid än vad är nödvändigt.

\picture{images/synaps.png}[0.5\columnwidth]


Med kännedom av synapsernas funktion blev ledande tanken bland forskare att låga serotoninhalter kunde kompenseras med att hålla serotonin längre mellan nervcellerna. Forskningens focus blev därmed att störa serotoninpumpens funktion med något lämpligt ämne. Det var önskvärt att utveckla läkemedel som inte påverkade andra signalämnespumpar, utan endast serotoninpumpen, s.k. selektiva serotoninåterupptagshämmare (selective serotonin reuptake inhibitor, SSRI). En svensk forskargrupp var år 1969 de första att utveckla en SSRI och bevisade dess effekt i behandling av depression.1 Några år senare utvecklades fluoxetin, bättre känd under varunamnet Prozac, en liknande SSRI som godkändes för bruk år 1987 av FDA. Bruket av antidepressanter, till vilka SSRI hör, har ständigt ökat i hela världen. År 2000 var konsumtionen i Finland 36 dagliga doser per 1000 invånare (DDD), medan år 2011 hade siffran fördubblats till 70 DDD. Medeltalet för de undersökta länderna var 31 DDD år 2000 och 56 DDD år 2011.3

På senare år har det blivit uppenbart att depression inte orsakas av serotoninbrist. Om antidepressanter fungerar, så är det inte för att de ökar serotoninhalten. Detta har lett till att bruket av antidepressanter har ifrågasatts.4 Situationen kan liknas med att Burana botar huvudvärk, men inte på grund av att kroppen har brist på Burana. Är det etiskt rätt att behandla patienter med läkemedel vars verkningsmekanism är okänd?

Situationen blev avsevärt mer komplex år 2006 då det upptäcktes att det finns två olika pumpar, serotonintransportprotein (SERT) och cellmembranmonoamintransportprotein (PMAT). SERT har länge varit känd och SSRI utvecklades för att störa denna, medan PMAT var en ny upptäckt. Serotonin binder sig ungefär 100 gånger bättre till SERT än till PMAT. Detta kan ge illusionen om att SERT ansvarar för största delen av upptagningen av serotonin. Faktum är att PMAT bidrar till upptaget med en avsevärd del, då den transporterar mycket fortare än SERT. Många SSRI blockerar även PMAT, men detta kräver mycket högre doser än vad som normalt används för depressionsbehandling.5

Vad lär vi oss om detta? Hjärnan är ett mycket komplext organ, som vi än idag inte fullt förstår. Upptäckten av PMAT-pumpen är ett bra exempel hur vår förståelse om en viss funktion, i detta fall upptagningen av serotonin, kan totalt förändras. En upptäckt kan ifrågasätta årtionden av medicinsk praktik, samt ställa läkemedelsindustrin i ett tvetydigt ljus. Kan man marknadsföra en produkt om den ''fungerar'' men man vet inte hur?

\end{artikel}



\picture{images/shanghai.jpg}[5.5cm][caption text]
\newpage



\begin{artikel}{En skål för våldtäkt}{Kadri}

I våras föreslog en gäst under Spektrums sits en skål för våldtäkt: ''Otetaan raiskaukselle!''. 
Det gjorde mig glad och stolt att ingen närvarande spektrumit besvarade hans skål. Skålen var ändå alldeles otänkbar och oförglömlig eftersom skålaren själv var en studerande i Gumtäkt.

Vid slutet av oktober i år kom MIT fram med resultat av sin undersökning om sexuellt övergrepp och sexuella trakasserier vid universitetet. Som en ansträngning att sätta stopp för sexuellt övergrepp på universitetsområden över hela Förenta Staterna, där 19\% av de kvinnliga studerandena erkänner att de har erfarit sexuellt övergrepp, har amerikanska undervisningsministeriet förpliktat högskolorna att rapportera statistiken om det. MIT:s undersökning är den grundligaste av alla publicerade undersökningar hittills. Den visar att det finns grundexamensstuderanden som inte vet vad sexuellt övergrepp innebär. Över 17\% av kvinnorna och 5\% av männen svarade att de hade erfarit oönskad sexuell kontakt i samband med fysiskt våld, fysiskt hot eller oförmogenhet att ge sitt samtycke. Ytterligare 12\% av kvinnorna och 6\% av männen svarade att de hade erfarit oönskad sexuell kontakt utan våld, hot eller oförmögenhet, vilket också kan utgöra sexuellt övergrepp beroende på omständigheterna. Bara 11\% av kvinnorna och 2\% av männen svarade att de hade våldtagits eller anfallits sexuellt. Samma sak gällde sexuellt trakasseri, dubbelt fler studenter anmälde att de hade hört olämpliga sexuella yttranden eller fått olämpliga meddelanden än att de hade erfarit sexuellt trakasseri.

Gällande sexuellt övergrepp enades många studenter med påståenden som rättfärdigade gärningsmannen och lade skulden på offren, t. ex. att offern inte vägrade klart nog.

Det syns att en stor del av de sexuella övergrepp som äger rum, sker eftersom ingen vet vad som klassificeras som samtycke till sexuell kontakt.

En annan amerikansk undersökning visar stora skillnader mellan hur kvinnor ger sitt samtycke och vad män tolkar som samtycke. 50\% av de heterosexuella kvinnostuderandena sade att de ger språkligt samtycke och heterosexuella manliga studeranden antog att 60\% av kvinnor ger sitt samtycke med kroppsspråk, medan bara 10\% av kvinnor faktiskt ger det med kroppsspråk. De är stora missförstånd med grova följder. Enligt samma undersökning är 63\% av männen våldtäktsmän: 27\% svarade att de får samtycke genom att befalla kvinnor att ha sex med dem, 14\% säger något i likhet med "låt oss ha sex!" och före hon kan säga något tar hennes byxor av henne, 13\% låtsas att sexuellt umgänge hände av misstag. Endast 22\% av männen sade att de får samtycke genom att fråga efter det.

\toppicture{images/kadri.jpg}[5cm][caption text]

Helsingfors Universitet är inte obekant med sådan strid. Ifjol publicerade Esitisle en oofficiell undersökning om sexuell kontakt på Gumtäkts guliskryssning. En kille och två tjejer anmälde till undersökarna att de hade haft sex av misstag, vilket kan innebära att de inte hade gett sitt samtycke och var följaktligen våldtagna.

I oktober läste jag en artikel om samtycke och var förbluffad över hur upplyst och revolutionerande den var. Sedan märkte jag att den hade skrivits år 1993. Artikeln berättade att sedan år 1990 krävs det i Antioch College i Ohio klart samtycke i ord före varje ny nivå av intimitet. De nya reglerna ledde till att kvinnliga studenter blev rättframmare om vad de ville och beteendet av manliga studenter blev mindre myndigt.

Antiochs riktlinjer blev mycket förlöjligade då, men nu har det uppstått en rörelse gällande klart sexuellt samtycke som svar till omfattande campusvåldtäkt i Förenta Staterna. California är i främsta stridslinjen av omdefinieringen av samtycke genom en lag som förpliktigar högskolor att lära studenter att det krävs ömsesidigt samtycke före varje sexuell kontakt, annars förlorar skolan statligt ekonomiskt bidrag. Många högskolor har introducerat krav av samtycke på egen hand, även  i  andra delar av USA.

Det är nämnvärt att MIT:s magister- och forskarstuderande visade bättre uppfattning om sexuellt övergrepp än grundexamenstuderandena och sexuella övergreppet själv var inte lika rådande bland dem. Det finns dock undantag som visar att bildningsnivå inte nödvändigtvis är något förebud av passande sexuellt förfarande. Mest anmärkningsvärt vid Helsingfors Universitet är en professor som avskedades i år på grund av bland annat sexuella trakasserier, efter vad som uppgivits.

Det finns ett överflöd av exempel på att alltid när män försöker skydda sitt levebröd eller sin sexistiska kultur, där allt är tillåtet, mot kvinnor tar de sin tillflykt till våldtäkt, hot och skämt om våldtäkt eller annat slags sexuellt trakasseri. Många som deltar i det offentliga livet får hatmejl, men det är bevisat att de hot som kvinnor brukar få är på en ny nivå jämfört med de hot som män får och omfattar ofta sexualbrott. GamerGate är bara det senaste exemplet på det här fenomenet. Det är därför en skål för våldtäkt ger kalla kårar uppefter ryggen på mig.

Det är ytterst passande att försök till att inkludera fler kvinnor i naturvetenskap och datavetenskap händer samtidigt som uppdatering av sexuellt samtycke. De båda strävandena riktar sig till objektifiering av kvinnor, vilket det inte finns rum för vid universitetet och i studentföreningar.

\end{artikel}


\begin{artikel}{Problemlösning – \\ sagan om att overkilla}{Jeremias}
\begin{twocolumns}
  Mycket av det som görs i Gumtäkt och speciellt i Exactum handlar om någon form av problemlösning. Redan under grundkursen i Analys presenteras man varje vecka med åtminstone sex problem man förväntas kunna lösa. Problemen förföljer studerandena genom hela kandidat-,  magisterexamen och för vissa även in i forskningsvärlden. Ofta kritiseras dessa problem; jag är nästan säker på att så gott som alla som studerat matematik eller data har i något skede av studierna tyckt att problemen som skall lösas är helt för artificiella, eller abstrakta för att vara intressanta: vem bryr sig om varför man inte får dela med $ 0$? För att hålla kvar de få matematikstuderanden vi har tänkte jag i denna text dela med mig ett exempel av applicering av abstrakta problemlösningstekniker för att åstadkomma saker som på riktigt spelar någon roll. Som huvudpersoner i historien fungerar tre namnlösa spektrumiter som vi kallar för Lasse, Peter och Jeremias, alla med en matematisk/datavetenskaplig utbildning.

  \columnpicture{images/very-hard-riddles.jpg}[0.7\columnwidth]
  Varje problemlösning börjar med ett problem. I vårt fall började vi från följande scenario:
  \textit{Tänk dig att du och två av dina vänner vid en given tidpunkt har fått en lapp med gåtor vars lösningar är sex olika barer i Helsingfors. Er uppgift är, att snabbare än 200 andra grupper på 3 människor, ta er till alla av dessa sex barer, inmundiga 1.5l öl på varje ställe, sedan ta er till Otnäs och inmundiga 1.5l öl till. Som färdmedel får endast allmäntransport användas.” För enkelhetens skull kommer jag från och med nu kalla detta scenario för ''Problem Y''.}

Problemlösning i sig är ett forskningsområde inom kognitiv psykologin. Inte oväntat så verkar det inte finnas ett absolut svar på frågan "hur skall man göra för att lösa problem?". Det finns dock vissa standardtekniker för problemlösning som dyker upp i olika undersökningar. Den man ofta börjar med inom matematik och datavetenskap kallas abstraktion. Abstraktion går ut på att exakt definiera vad man försöker åstadkomma, samt identifiera vilka delar av den information man har som behövs för att utveckla en lösning. Dessutom brukar man under abstraktion ofta identifiera begränsningar som sätts på de metoder man kan använda i sin lösning. Ett sätt att beskriva problemet efter abstraktion är genom att använda orden ''INPUT'' för att beskriva den (viktiga) informationen man har, ''OBJECTIVE'' för att beskriva det man försöker åstadkomma och ''CONSTRAINTS'' för att beskriva de begränsningar ens lösning måste följa. Då det kommer till problem Y lyckades vi inte komma på ett sätt att använda (lagliga) datavetenskapliga/matematiska metoder för att hjälpa oss med inmundigandet av $7 \times 1.5$ liter vätska eller påverka de $200$ andra lagen. Därför bestämde vi oss för att ignorera dessa delar av Y och fokusera på lösningen av gåtorna samt bestämmandet av rutten att ta. Som begränsningar för problemet identifierade vi faktumet att rutten måste endast bestå av allmäntransport. Efter abstraktion såg Y alltså ut såhär:

INPUT: startställe, gåtor till 6 barer

OBJECTIVE: Kortast möjligast tid för att lösa gåtorna, ta sig runt till barerna och sedan ta sig till Otnäs.

CONSTRAINTS: Endast allmänna färdmedel tillåtna.

Efter abstraktion är det sedan dags att börja söka en lösning. En inom psykologin identifierad och inom Exactum utlärd teknik för problemlösning kallas divide and conquer. Idén är att spjälka upp ett komplext problem i mindre delar i hopp om att delarna skall vara lättare att lösa. Ofta strävar man till att delproblemen är oberoende av varandra, då kan nämligen alla delproblem lösas samtidigt.  I bästa fall kan optimala lösningarna av delproblemen sen kombineras för att åstadkomma en optimal lösning på ursprungliga problemet. Det finns ett ganska naturligt sätt att dela in Y: lösningen av gåtorna samt bestämmande av rutten. Visserligen är dessa två problem inte riktigt oberoende, man kan inte bestämma rutten innan man har löst alla barer. Vi bestämde oss ändå för att dela in Y såhär i hopp om att gåtlösaren skulle bli tillräckligt effektiv för oss att ha någon nytta av rutlösaren.  Olyckligt nog visade det sig att gåtlösning högst troligen är för svårt för oss att automatisera. Det finns en dator vid namnet Watson som har  lyckats slå  de bästa människorna på Jeopardy, men att åstadkomma detta tog 7 år för IBM:s topp ingenjörer. För att vara effektiv måste Watson dessutom köras parallellt på 70 datorer samtidigt. Eftersom vi inte kände för att börja bära omkring på 70 datorer samt inte hade 7 år att sätta på projektet, var det bästa vi kunde hoppas på ett hjälpmedel för gåtlösning, inte en automatisering. Det vill säga, vi skulle inte kunna kombinera lösningarna av delproblemen för att få en optimal lösare av Y. Liknande fenomen sker ofta inom datavetenskap/matematik. Ifall ursprungliga problemet man löser är ”tillräkligt svårt” kan man inte kombinera dellösningarna för att åstadkomma en optimal lösning. Men ofta visar det sig att den (lite sämre) lösningen man åstadkommer ändå är ”tillräkligt bra”, vilket också kommer att vara fallet med Y. Resten av denna text kommer att koncentrera sig på rutt-lösnings problemet: "givet 6 barer samt en startposition hitta snabbaste sättet att ta sig till alla 6 barer och Otnäs med hjälp av allmäntransport". Detaljerna av gåtlösaren kommer eventuellt att publiceras i ett senare skede (läs: ifall jag inte får en bättre idé för en höstartikel).

Då man söker efter en lösning till ett ovanligare problem är det alltid inte klart att det man försöker göra är svårt. Detta kan vara bra att ta reda på;  det är rätt så onödigt att utveckla hemskt komplicerade lösningar till problem som egentligen (bevisligen) är simpla. ”Hur svårt kan det vara?” var alltså nästa fråga vi ställde oss. En systematisk analys av olika problems svårighet är ett delområde inom komplexitetsteori och är ett forskningsområde för sig. Tekniska detaljer åsido så går analyserna av problemsvårigheter ofta ut på att man förliknar sitt problem med andra kända problem som man vet är svåra/lätta. Ofta är dessa kända problem relativt abstrakta, mest p.g.a. att det skall vara enklare att se likheterna mellan dem och det egna problemet. Då det kommer till vårt problem: ”givet en start position, 6 barer samt ett mål, hitta snabbaste rutten till alla” finns det ett relativt känt problem som man enkelt kan jämföra med, nämligen traveling salesman problem (TSP). I ursprungliga formuleringen av TSP ingår en försäljare som skall besöka alla städer i Tyskland och/eller Schweiz (ursprunget är oklart) i hopp om att sälja sina varor i var och en. Uppgiften är att försöka hitta den kortaste möjliga rutten att göra detta. I en mer abstrakt formulering av TSP  har man en matematisk graf bestående av något antal noder och kanter där varje kant har en given längd. Målet är att hitta den kortaste möjliga rutten genom grafen som besöker varje nod åtminstone en gång. Ruttlösningen för problem Y kan ses som travelling salesman där noderna i grafen är de 8:a olika platserna som skall besökas och längden på en kant mellan två noder är tiden som det tar att ta sig mellan dem med allmäntransport. Denna jämförelse mellan problemen är tillräcklig för vårt behov, men inte helt fullständig, vilket vi kommer att märka lite senare.

\columnpicture{images/tsp1_s.jpg}[0.7\columnwidth]

Efter att man ”känt igen” sitt problem kan man sen dra slutsatser på huruvida man kan förvänta sig att det går att komma fram till en lösning som fungerar inom en rimlig tid. I vårt fall skulle ju en rutt-lösare inte hjälpa hemskt mycket ifall det skulle ta 5 timmar att använda den. Olyckligt nog är traveling salesman ett NP-svårt problem. Igen tekniska detaljer åsido, betyder detta att det i allmänna fallet knappast finns ett effektivt sätt att lösa TSP. Tur i oturen kommer dessa teoretiska begränsningar emot endast då man ökar mängden noder i grafen. Med andra ord så går det troligtvis inte att koda en effektiv ruttlösare för att komma till 100 barer. Men detta var ju inte vad vi var ute efter; våra levrar skulle ändå  inte stå ut med så mycket öl. Att lösa rutten till 6 barer borde däremot inte vara en utmaning, ens åt Lasses miniläppäre. Vi kunde alltså fortsätta på projektet trots vissa teoretiska begränsningar.

Efter all förberedning visade sig själva kodandet av ruttlösaren var relativt lätt. Efter lite undersökning märkte vi nämligen att HSL bjuder på ett gratis gränssnitt till sin reittiopas, vilket i princip betyder att programmerandet av en normal reittiopas skulle (nästan) kunna vara en uppgift under ohjelmoinnin jatkokurssi. Inspirerade av detta samt de tidigare teoretiska funderingarna döpte vi vår ruttlösare till Drunken Salesman Opas (DSOPAS). Under kodandet av DSOPAS stötte vi egentligen endast på ett till problem som förhindrade oss från att lösa rutten optimalt. Problemet har att göra med jämförelsen av vårt problem och TSP. Då det kommer till allmän transportation kan det ju nämligen hända att tiden det tar att komma från bar X till Y är olika beroende på vilken tid på dagen man reser. Mera matematiskt formulerat så är längden av varje kant i grafen inte konstant. Detta betyder att för en optimal lösning av vårt problem borde reittiopas "frågas" skilt för vikten av alla kanter i varje rutt möjlighet som undersöks. I en rutt ingår $7$ kanter, allt som allt finns det $6! = 720$ olika ruttmöjligheter. Totalt skulle algoritmen vid varje körning vara tvungen att fråga reittiopas efter längden av $5040$ kanter. Detta skulle annars inte vara ett problem, men HSL tillåter endast $5000$ frågor i timmen. Så fast det skulle vara helt möjligt att koda DSOPAS såhär, skulle det ta minst en timme att köra den. För att komma över detta problem bestämde vi oss för att än en gång approximera optimala lösningen genom att helt enkelt skita i att tiderna eventuellt ändrar. Den slutgiltiga DSOPAS börjar alltså med att fråga reittiopas efter tiden det tar att röra sig mellan alla 8 platser (6 barer, start och mål) vid en fixerad tidpunkt ($\frac{6 \times 5}{2}$ kanter mellan barerna, $6$ från startpunkten till de olika barerna och $6$ från de olika barerna till målet, allt som allt $27$). Efter det löser DSOPAS den kortaste rutten i denna ”approximeringsgraf” med hjälp av en sofistikerad algoritm känd som ”prova alla möjligheter”. Under lösningen behöver DSOPAS inte fråga reittiopas alls. Slutligen väljer DSOPAS ut de 5 snabbaste rutterna, vilka sen körs på nytt genom reittiopas för att få egentliga tiderna det tar att använda var och en. Dessa fem rutter visas sen åt användaren. Att fråga reittiopas om tiderna för en fixerad rutt kräver ju alltså $7$ frågor. Allt som allt $27 + 5 \times 7 = 62$ frågor,  betydligt bättre än ursprungliga $5040$.

\columnpicture{images/dsopas2.png}[0.8\columnwidth]

Efter att man löst sitt problem så återstår ett sista steg: verifikation av lösningen. Fastän vi kunde testa DSOPAS på påhittade rutter, var vi tvungna att vänta på årets Ylonz (som de flesta av er säkert har fattat är det Ylonz detta handlar om) innan vi kunde veta hur bra det skulle fungera på riktigt. (För att vara helt ärlig blev väntandet inte hemskt långt, sista buggarna i DSOPAS fixades onsdagen innan Ylonz lördagen). Jag kommer inte att prata så hemskt mycket om hur det gick, många av er vet det redan och för resten kan ju nämnas att vårt lag hette ”<script>alert (‘TechSupported’);</script>” samt att resultaten finns här. Jag påstår inte att DSOPAS var ända orsaken till att resultaten ser ut som de gör, men jag kan med största säkerhet säga att den tillsammans med vårt gåtverktyg nog hjälpte betydligt.

Det fanns en bar i årets Ylonz rutt som fungerar väl som exempel för nyttan vi hade av båda verktygen:  Park hotell. Då vi, baserat på bilden av ett monopolhotell, slog in ”hotell” i gåtlösaren var ”Park hotell” det enda alternativet vi fick ut. Dumma som vi var trodde vi inte på vårt eget program, men fortsatte istället fundera.  Fem minuter senare hade vi löst resten av tipset och kom fram till att gåtlösaren hade rätt från början. Senare, då vi kom fram till baren, visade sig att den långa kön skulle förhindra oss från att hinna på den bussen som DSOPAS hade bestämt åt oss tidigare. Då, detta under tidigare år skulle ha lett till en massa spekulationer på vad vi borde göra näst, kunde vi nu helt enkelt mata in de återstående barerna i DSOPAS på nytt. Den nya rutten vi fick av vårt program visade sig sist och slutligen vara lika snabb som den första vi hade fått.

Den här texten har egentligen två syften. Den ena är ett exempel på hur ”viktiga problem”  kan modeleras och lösas genom abstrakta vetenskapliga metoder. Den andra kan sammanfattas i form av en utmaning. Åtminstone 1/3 av laget TechSupported kommer inom snarast framtid att försvinna utomlands och det verkar som vi inte kommer att ha möjlighet att tävla i Ylonz med samma laguppsättning igen. Men det är inget i användningen av DSOPAS samt gåtlösaren som specifikt kräver att man kan programmera. Dvs. om det är något annat Spektrum lag som känner sig intresserade av att pröva hur långt man kommer i Ylonz med lite teknikhjälp är det bara att ta kontakt. Som sagt påstår vi inte att programmen automatiskt leder till framgång, men får man tipsen löst snabbt och är färdig att springa, har man alla möjligheter att klara sig mycket väl. Spektrum har en fin vana att klara sig väl i Ylonz, en vana som började länge före oss. Vad tycker ni, nog skall vi väl hålla segern inom Spektrum några år till?
\end{twocolumns}

\end{artikel}


\begin{artikel}{Ren matematik, datamatematik \\ och hur man kan lösa det olösliga}{Sebbe}
  En betydande del av den nutida forskningen inom matematik använder datorer som hjälpmedel. Naturligtvis är användningen av datorer mest utspridd i tillämpad matematik och i statistik (bl.a. i form av Monte Carlo-metoder), men dataprogram kan även användas för att bevisa satser som känns väldigt teoretiska. Det mest välkända exemplet lär vara beviset på fyrfärgssatsen.

Datorer har dock ett antal begränsningar: man kan mekaniskt, genom att gå igenom fall ett för ett, bevisa något endast för ett ändligt antal fall. Beräkningar gjorda av en dator är dessutom inte exakta i alla fall. Detta beror på att det i de flesta fallen finns endast en begränsad mängd minne reserverat för decimaltal, och därmed tappar man i vissa fall precision då man utför beräkningar. Oftast är detta ej önskevärt, men i vissa specialfall kan man faktiskt utnyttja denna brist för att göra beräkningar som tekniskt sett inte borde vara möjliga!

Tänk dig att vi skulle i något sammanhang ha behov av en funktion som tar in två reella tal och ger som resultat dess summa, förutom om talen är lika. Ifall talen är lika ger den ut det ena av talen (vilket av dem spelar ingen roll eftersom de är lika). Uttryckt matematiskt är vi alltså intresserade av funktionen $f\colon \mathbb{R}^2 \rightarrow \mathbb{R}$, där

    $f(x, y) = x + y$ ifall $ x \neq y$, och
    $f(x, y) = x$ ifall $ x = y$.

På en dator kan ovanstående konstruktion i vanliga fall lätt implementeras med hjälp av en if-sats, men låt oss nu anta att vi av någon orsak inte kan använda oss av konditionaler. Istället måste vi uttrycka ovanstående funktion med hjälp av endast de fyra grundräknesätten. Först skulle vi säkert försöka skapa en multiplikation med noll i något skede för att få en av termerna att försvinna i fallet där de båda är lika:

$ x + (x-y)\cdot y \quad \: (1)$.

Detta fungerar ifall $ x = y$, men om $ x \neq y$ får vi $ x + xy - y^2$, vilket vi inte ville. Vi borde alltså i fallet $ x \neq y$ lyckas få den tillagda överloppstermen $ x-y$ att försvinna. Ett sätt är att istället använda sig av formeln

$ x + \frac{x-y}{x-y} \cdot y \qquad \: (2)$.

Ifall $ x \neq y$, så ger ovanstående formel det korrekta resultatet $ x + y$. Tyvärr så fungerar den ej i fallet $ x = y$, eftersom vi då stöter på problemet att dividera med noll.

Eftersom de första, mest uppenbara försöken att skapa formeln misslyckades, kan det löna sig att fundera en stund över orsaken på varför vårt tillvägagångssätt ej fungerade. Vi märker att funktionen $ f\colon \mathbb{R}^2 \rightarrow \mathbb{R}$ är definierad för alla par av reella tal. Därmed kan vi inte i något skede dela med ett uttryck som kan bli noll, eftersom vi då får till stånd ett par av reella tal där formeln ej är definierad.

Ifall vi undersöker funktionen $ f$ närmare märker vi även att den inte är kontinuerlig i hela $ \mathbb{R}^2$. Problempunkterna är punkterna av typen $ (a,a)$ där $ a \neq 0$. Om vi t.ex. slår fast x-koordinaten att vara 2, ser vi att det för gränsvärdena gäller att $$ \lim\limits_{y \to 2^{+}} f(2,y) = \lim\limits_{y \to 2^{-}} f(2,y) = 4,$$ men att $ f(2,2) = 2$. Vårt mål var att beskriva funktionen $ f$ endast med hjälp av de fyra grundräknesätten. Det är välkänt att funktioner som är uppbyggda endast med hjälp av de fyra grundräknesätten är kontinuerliga i hela sin definitionsmängd. Därmed är det omöjligt att beskriva funktionen $ f$ endast med hjälp av grundräknesätten eftersom funktionen inte är kontinuerlig!

Det verkar alltså som vi skulle ha kört fast, eftersom det rent matematiskt är omöjligt att lösa problemet, givet de verktyg vi har. Detta är dock ett av de fall där vi kan utnyttja de begränsningar som datorer ställer på beräkningar av decimaltal. I vanliga fall beskrivs decimaltal av en dator på följande sätt:

$ signifikand \cdot bas^{exponent}$.

Eftersom varje decimaltal ges en konstant mängd utrymme i datorns minne, finns det en gräns på storleken på signifikanden (dvs. hur många signifikanta siffror talet kan ha) och även en gräns på hur stor exponenten kan vara. Detta leder till att datorn kan representera tal som är nära noll mycket noggrant, medan precisionen (absolut sett) blir allt mindre ju större talet är. Ifall vi som exempel slår fast basen att vara $ 10$ och antalet signifikanta siffror att vara $ 10$ och jämför exponenterna $ -20$ och $ 20$, ser vi att vi kring talet $ 10^{-20}$ kan representera skillnader av storleksordningen $ 0.000000001 \cdot 10^{-20} = 10^{-11}$, medan vi kring talet $ 10^{20}$ endast kan representera skillnader av storleksordningen $ 0.000000001 \cdot 10^{20} = 10^{11}$! Därmed om vi t.ex. skulle göra beräkningnen $ 10^{20} + 1$ skulle resultatet vara $ 10^{20}$, eftersom vi ej kan representera tal så noggrant kring $ 10^{20}$.

Hur kan vi då utnyttja detta fenomen för att lösa vårt ursprungliga problem? Vi hade tidigare skapat formeln $ (2)$, som fungerar bra förutom att vi kan tvingas dela med noll. För att undvika division med noll kan vi modifiera formeln på följande sätt:

$ x + \frac{x-y}{(x-y) + \varepsilon} \cdot y \qquad \: (2')$,

där $ \varepsilon$ är ett till absolutbeloppet mycket litet positivt tal (storleken på talet beror på representationen av decimaltalen). Matematiskt sett gäller naturligtvis inte att $ (x-y) + \varepsilon = x-y$, men p.g.a. bristerna i decimaltalsberäkningen kommer likheten att gälla ifall $ x$ och $ y$ är olika och är tillräckligt stora. Kvoten $ \frac{x-y}{(x-y) + \varepsilon}$ är alltså $ 1$ ifall $ x \neq y$, och därmed uppfyller formeln $ (2')$ kravet att $ f(x,y) = x + y$ om $ x \neq y$.

Å andra sidan ifall $ x = y$, så är naturligtvis $ x - y = 0$ och därmed är $ 0 + \varepsilon = \varepsilon$, eftersom vi kan representera tal noggrant kring noll. Därmed undviker vi division med noll och kvoten $ \frac{x-y}{(x-y) + \varepsilon}$ blir 0. Alltså gäller det för formeln $ (2')$ även att $ f(x,y) = x$ ifall $ x = y$.

Vi har alltså lyckats trolla fram en formel som med hjälp av datorberäkningar löser ett matematiskt sett omöjligt problem. Dock har formeln sina begränsningar; t.ex. får talen $ x$ och $ y$ ej vara för små, för då skulle additionen av talet $ \varepsilon$ påverka det slutliga resultatet som formeln $ (2')$ ger. I de flesta fallen är precisionsproblemen i decimaltalsräkningen något man försöker undvika (se t.ex. Wikipedia-artiklen om ämnet), men i vissa fall kan man faktiskt dra nytta av dem. Det är bra att komma ihåg att begränsningar som datorer ställer inte alltid är endast negativa; i vissa fall kan vi p.g.a. begränsningarna lösa problem som annars vore olösliga.

Jag vill till sist tacka Lasse för idén till artikeln! Den som är intresserad att lära sig mer om ämnet kan börja t.ex. med Wikipedia-sidan IEEE floating point.

\end{artikel}

\begin{artikel}{Dörrkodsproblemet}{Sebbe}
\begin{twocolumns}
  \textbf{En skrämmande tanke}

Föreställ dig följande scenario: Du är på väg till århundradets fest som förstås ordnas i Majstranden. Då du sitter på bussen märker du att du har glömt telefonen hemma, men du låter inte en sådan liten motgång sänka humöret. När du äntligen anländer till Majstranden och skall slå in dörrkoden märker du att du inte kommer ihåg den! "Inget problem", tänker du, för någon måste ju snart komma ut och så kan du slinka in samtidigt. Efter att ingen setts till på en kvart fattar du plötsligt att alla andra redan måste vara på festen (det är ju århundradets fest vi talar om), så det är klart att ingen är på väg ut på länge!

"Ingen orsak att panikera", tänker du, "detta problem går nog att lösa". Du samlar ihop allt vad du vet om problemet: koden är fyra siffror lång (för alla koder är ju det), och från manicken du slår in den på ser du att den kan bestå av siffror från 0 till 9. Du beräknar snabbt att det finns totalt $ 10^4 = 10000$ möjliga koder, så ifall du prövar dem alla, måste dörren öppnas förr eller senare. För inmatandet av alla koder måste du då slå in $ 4 \cdot 10000 = 40000$ siffror.

\columnpicture{images/dorrkod.jpg}[0.4\columnwidth]


Efter att du funderat på problemet en stund till, så kommer \sout{någon och öppnar dörren} du ihåg att någon sagt åt dig att dörrkodsläsare brukar fungera lite konstigt. De kräver nämligen inte att man håller en paus efter att man slagit in de fyra siffrorna förrän den kollar ifall koden är korrekt. För att dörren skall öppnas räcker det att man i något skede slår in de fyra korrekta siffrorna i rad. Det har alltså ingen skillnad fast man matar in felaktiga siffror i början så länge den korrekta koden slås in i något skede. Ifall du slår in fem siffror har du alltså matat in två olika fyrsiffriga koder! Den första består av siffrorna 1-4, och den andra av siffrorna 2-5.

Du fattar direkt att för att alla koder skall finnas någonstans i den sekvens du slår in, måste du knappast mata in 40000 siffror, eftersom koderna överlappar varandra. Vilken strategi skulle det löna sig att använda för att få inslaget alla koder med så få siffror som möjligt? Naturligtvis lönar det sig att undvika upprepning av koder, alltså vill du i varje skede slå in en sådan siffra som skapar en fyrsiffrig kod som du inte matat in tidigare.

\textbf{Simplifikation av problemet}

För att simplifiera problemet rycker vi oss för en stund bort från vårat scenario och funderar i stället på en dörrkodsläsare som endast har siffrorna 0 och 1 och kräver en tvåsiffrig kod. Det finns alltså $ 2^2 = 4$ olika koder, nämligen koderna $ 00$, $ 01$, $ 10$ och $ 11$. I vilken ordning skulle man slå in siffror i detta fall för att med så liten möda som möjligt få upp dörren? En möjlighet är att slå in sekvensen $ 11001$. Den innehåller alla koder:
\begin{align*}
  \boldsymbol{11}001 \\
  1\boldsymbol{10}01 \\
  11\boldsymbol{00}1 \\
  110\boldsymbol{01}   
\end{align*}
Är det möjligt att hitta en kortare sekvens som innehåller alla koder? Genom att analysera listan ovan ser vi fort att det inte kan gå. För att visa detta, antag att $ L$ är en sekvens av siffror 0 och 1. Ifall den innehåller alla fyra olika koder av längd två, måste var och en av koderna starta vid olikt index i sekvensen. Därmed kan den sista koden starta som tidigast vid index 4, och eftersom den är av längden två, måste sekvensen $L$ vara minst fem siffror lång. Detta betyder att är sekvensen $ 11001$ den kortaste möjliga lösningen. Dock finns det andra lösningar av längden fem också, t.ex. $ 00110$.

\picture{images/01dorrkod.png}[0.5\columnwidth]
Dörrkodsläsare, modell datalog (Patent pending)

Hur skapades då sekvensen $11001$? En möjlig strategi är följande: börja med den största koden (i detta fall $ 11$), och välj sedan den lägsta möjliga siffran som skapar en ny kod till sekvensen, dvs. en kod som inte från tidigare finns i sekvensen. Ifall detta alltid lyckas, kommer vi att få till stånd en möjligtvis kort sekvens, eftersom den inte upprepar några koder.

Ifall vi använder denna strategi i vårt ursprungliga scenario, kommer vi att få till stånd en sekvens av längden $ 10^4 + (4-1) = 10003$ (börjandes med siffrorna $ 9999$) som innehåller alla koder, en klar förbättring till att vara tvungen att mata in 40000 siffror!

\textbf{Det allmänna problemet}

Den stora frågan är följande: Kan vi för alla kodlängder $ k$ och sifferantal (dvs. baser) $ b$ alltid skapa en sekvens som inte upprepar någon kod två gånger? Svaret är kanske aningen överraskande ja. Beviset i sin helhet, till vilket det finns en länk i slutet av artikeln, är lite för invecklat för att tas upp i denna artikel, men det bygger på följande observation: Antag för enkelhetens skull att vi vill skapa koder av längd 4 som består av siffrorna 0, 1, 2 och 3. Ifall vi har någon kod, t.ex. $ 0233$, kan efter denna kod i sekvensen följa endast en kod som börjar med siffrorna $ 233$ och slutar med en av siffrorna 0, 1, 2 eller 3. Detta gör att sekvensen blir mycket symmetrisk, eftersom det efter någon av koderna $ 0233, 1233, 2233, 3233$ endast kan följa någon av koderna $ 2330, 2331, 2332, 2333$.

\textbf{Tillämpning till scenariot}

Smart som du är kom du på denna lösning till problemet på säg...22 minuter. Du började sedan knacka in siffrorna, och eftersom du har ett utomordentligt minne kunde du hålla reda på vilka koder du redan slagit in. Efter en dryg timme lyckades du därmed få upp dörren, och kvällen var räddad. Snäll som du är skapade du åt dina vänner en textfil med den kod som behövs för att öppna vilken som helst dörr med fyrsiffrig kod i bas 10!

\textbf{Till sist}

Ett stort tack till Lasse och Jeremias för hjälpen med detta problem!

\end{twocolumns}
\end{artikel}



\begin{svammel}

	\begin{svammelpage}
	\SvammelLeft{Jes, nu har jag massa karlar som\\mina slavar! Och Fanny.}{Camilla}
	\SvammelRight{Bieber är cool!}{Emppu}
	\SvammelLeft{Salmari är väl inte öl?}{Jonas}
	\SvammelRight{Det värsta som nog kan hända är att man\\sku hamna sätta boxers på sig.}{Sebbe B}
	\SvammelLeft{Jag vill så fylla det där hålet med något!}{Johan}
	\end{svammelpage}
	
	\begin{svammelpage}
	 \SvammelLeft{Hallå? Jag hör inte dig. \\ Aj, du har inte svarat ännu.}{Toffe (på telefon)}
	 \SvammelRight{Får du upp den? \\Jo, jo… \\– en stund senare - \\Den stiger! \\Den brukar nog sjunka… \\Men titta – den stiger, den stiger! \\– ännu en stund senare - \\Mikka, du kan nog komma nu}{Micka, Toffe, Tingeling}
	 \SvammelLeft{I rumpan är awesome}{Sandra}
	 \SvammelRight{Hundrasex, det är som hundraklubben med...}{Jonas}
	 \SvammelLeft{Jag får inte in den i Niklas}{Sebbe F.}
	 \SvammelRight{Jag borde inte ha ätit 600g kött innan det här}{Toffe}
	\end{svammelpage}

	
\end{svammel}



%----------------------------------------------------------------------
%	This command prints the publisher information on one line with
%	all fields separated by 'bullets'.
%----------------------------------------------------------------------
\lastpage{0.9\textwidth}{images/Lalreklam.pdf}

\end{document}