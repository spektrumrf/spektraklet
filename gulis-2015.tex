\documentclass{spektraklet}

\begin{document}

%----------------------------------------------------------------------
%	arguments:
%		#1	publication number (eg. 1/2015, 2/2015, 3/2015 etc.),
%			the year is filled in programatically.
%
%		#2	the width of the cover image
%
%		#3	the path to the cover image
%
%		#4	optional argument with offsets for the cover image
%----------------------------------------------------------------------
\titlepage{Gulisnummer}{\paperwidth}{images/01dorrkod.png}[0, -1cm]
%\titlepage{1}{0.75\linewidth}{images/dog.jpg}



%----------------------------------------------------------------------
%	There are a couple of fields in the content page that is changable,
%	and some of them are mandatory. The default values are defined in
%	the 'spektraklet.cls' file.
%
%	Mandator:
%		\ChiefEditor{name}
%		\ManagingEditor{name}
%		\Editors{name1\\name2\\name3}	the last name shouldn't be followed by a \\
%
%	Optional:							Default value:
%		\PublicationInformation[text]		Àr ett språkrör för de som studerar - eller låter bli
%											att studera - matematik, fysik, kemi eller datavetenskap
%											på svenska vid Helsingfors universitet
%		\PublicationSupport[text]			Spektraklet får HUS-stöd för föreningstidningar
%		\PublisherName[text]				Spektrum rf
		\PublisherAddress[Exactum]				%Kemiska institutionens svenska avdelning
		\PublisherPostalOfficeBox[PB 68]		%PB 55
%		\PublisherPostalCode[text]			00014 Helsingfors universitet
%----------------------------------------------------------------------


\ChiefEditor{Celina Schröder}
\ManagingEditor{Sandra Kulla}
\Editors{
Jeremias Berg\\
Sebastian Björkqvist\\
Jere Mannisto\\
Johan Blomqvist\\
Kadri Kiilas\\
Fanny Bergström\\
Iris Wrede\\
Oskar Björkman
}
\CoverPageAuthor{Celina Schröder}


% Include the content page.
\contentpage




\begin{ledaren}{Celina}

%\wrappicture{images/celina.jpg}[4.5 cm][R]

Hej och välkommen blivande spektrumit!

Från Spektraklets sida vill jag önska dig välkommen till Helsingfors universitet och Spektrum. För att underlätta de första veckorna på Uni har redaktionen under sommaren sammanställt detta specialnummer av Spektraklet, som är specifikt riktat till nya studerande. Här kan du läsa om studierna, studielivet och Spektrum och kanske få en lite mer nyanerad blick över hur det är att studera i Gumtäkt.

Spektraklet är Spektrums medlemstidning som publiseras på bloggen www.spektrun.fi/spektraklet. Dessutom publiseras en e-tidning med terminens bästa artiklar på Spektrums webbplats två gånger per år. Redaktionen är alltid intresserad av nya medarbetare, så om du brinner för att skriva, teckna, fotografera, göra videon eller något annat kreativt är du välkommen med och ta i så fall kontakt med mig.

I detta nummer kan du läsa om bland annat vem som är med i styrelsen, vad tutorerna minns som det bästa från sitt gulisår och hur det kan vara att som finskspråkig komma till en svenskspråkig förening. Dessutom har vi gjort en medlemspresentation, så att du lättare ska kunna lära känna oss alla i Spektrum.

Jag hoppas att du ska få ett minnesvärt gulisår här i Gumtäkt och att vi ses både på gulisprogrammen och andra program som Spektrum ordnar senare i höst. Ta vara på ditt gulisår, för det flesta händer det bara en gång!

Till sist:

"\textit{Det är ytterst viktigt att inte helt staka ut sitt liv, att inte genomorganisera dagar och veckor. Det ät nödvändigt att ha luckor och öppningar för spontana aktiviteter, eftersom det är där vi öppnar oss för slumpens obegränsade möjligheter.}"  J. Holmes

\end{ledaren}





\begin{ordforandespalten}{Jonas}

\wrappicture{images/jonasordf.jpg}[4 cm][]

Sätt dig ner och ta ett djupt andetag. Stäng dina ögon och låt dina tankar flyta till uppfattningar om ditt liv före denna stund. Du har kanske avlagt din student- eller yrkesexamen, varit på utbyte, utfört din militärtjänst, eller njutit av ett mellanår.
 
Du har alltså upplevt en hel massa, men ingenting jämfört med detta. När du öppnar dina ögon kommer du att vara universitetsstuderande; ditt liv kommer att ändra dramatiskt. Så skulle jag skriva om det vore sant och om jag ville spänna på dramamusklerna. Steget in i universitetslivet innebär många förändringar och överraskningar, men inget så stort som överraskningen över att inte så mycket ändrar. Jo, du kommer att få ansvar för att dina studier framskrider som planerat. Du kommer att få helt ny frihet för att studera det du faktiskt gillar. Du får träffa nya människor och göra saker som du kanske inte tänkte att du skulle göra. Men allt detta uppenbarar sig endast om du själv bestämmer att det skall hända.
 
På universitetet finns flera studierådgivare som kan berätta åt dig exakt vad och när du skall studera. Universitetet har dig att göra en plan över dina inkommande studier och stöder dig genom tutorer och handledningsverksamhet. Ingen tvingar dig att delta i ämnesföreningar eller nationer och deras evenemang. Det är alltså fullt möjligt att fortsätta vidare från gymnasiet utan större förändringar. Men då går man enligt mig miste om det som, enligt mig, gör universitetet till en sån fin sak. Nämligen just att det är ett nytt och annorlunda steg i livet.
 
Var alltså inte rädd att välja ut just det som intresserar dig och våga bli expert på ditt område. Njut av att du får själv välja takten och studiestilen som passar dig och din egen inlärning. Och viktigast av allt, kom med och upplev föreningsverksamhet på bästa sätt. Upplev sitzer, spelkvällar, exkursioner, årsfester, salturer, filmkvällar, allmän samvaro och struntprat. När ni i framtiden tar ut er examen så kommer ni inte att komma ihåg föreläsningarna eller räkneövningarna, utan människorna och alla de roliga stunder ni har fått vara del av.
 
Öppna dina ögon (editor note: too corny) Välkomna till Spektrum och studielivet!

\end{ordforandespalten}





\begin{artikel}{Tutorpresentation}{Paul, Frans, Oskar och Fanny}



\wrappicture{images/Paul.jpg}[3cm][R]
\textbf{Paul Vuorela, matematiktutor}

Hejssan! Paul heter jag, är matematiker med datavetenskap, statistik och biovetenskaper som biämnen. Jag är intresserad av att i framtiden undersöka sjukdomar noggrannare med matematiska medel. Som mina intressen kan musik, god mat, jogging, samt hälsosam prokrastinering räknas. Inom Spektrum fungerar jag i år som värd, sångledare och programkomittémedlem. Jag kommer att vara aktivt med på Spektrums evenemang så vi möts där! Och om inte där, så stöter man på mig lätt i Spektrums kafferum på Uni, där jag ofta brukar spela Bang! med andra spektrumiter.

De tre bästa sakerna under mitt gulisår var:
\begin{enumerate}
\item Nya likasinnade vänner och andra bekantskaper.
\item Studierna: friheten och ansvaret med dem.
\item Mellanåret i milin som fick mig att längta tillbaka till studierna, och därmed bli aktivare.
\end{enumerate}


\wrappicture{images/Frans.jpg}[3cm][R]
\textbf{Frans Graeffe, fysiktutor}

Morjens, jag heter Frans och är en fysiker som helt enkelt bara 
vill förstå hur världen och dess fenomen fungerar, därav mitt ämnesval. Förutom fysik är jag intresserad av motion i olika former, vinylskivor samt äldre musik i allmänhet , Star Wars och Aku Ankka (obs, endast på finska!). Jag är aktiv i de flesta spektrala evenemang som ordnas samt sitter i årets styrelse. Jag har precis som många andra spektrumiter fastnat för det fenomenala Bang!-spelet, som jag gärna lär alla nya gulisar!

De tre bästa sakerna under mitt gulisår var:
\begin{enumerate}
\item Känslan av frihet av att flytta bort hemifrån och få själv ta ansvar för sina studier samt sitt boende.
\item Alla de äldre spektrumiterna som tagit emot mig till en fantastisk gemenskap där alla kan känna sig som hemma.
\item I sin korthet: det akademiska studielivet!
\end{enumerate}

\newpage

\wrappicture{images/oskar.jpg}[3cm]
\textbf{Oskar Koli, datatutor}

Halloj, jag heter Oskar och är andra årets datavetenskapstuderande. Jag började programmera som 13-åring, så jag är långt självlärd. Jag har jobbat på Rovio år 2012, men har sen dess bytt till ett annat spelföretag "Seriously", där jag jobbar 3 dagar i veckan vid sidan om studierna. Jag har inte varit aktiv inom Spektrum förrän jag nu blev tutor, så jag ser fram emot att lära känna Spektrum i takt med gulisarna!

De tre bästa sakerna under mitt gulisår var:
\begin{enumerate}
\item Friheten och ansvaret för sig själv och studierna
\item Att få studera ett ämne som verkligen intresserar
\item Människorna
\end{enumerate}


\wrappicture{images/Fanny.jpg}[3cm]
\textbf{Fanny Bergström, kemitutor och studiesekreterare}

Hejssvejs,
Jag är Fanny, är en analytisk kemist med matte som biämne. Som intressen kan man anse mig ha orientering, vardags kemi, Tyskland och öl. Jag sysslar vid Uni med allt möjligt; TvEx-koordninator, räkneövningsassistent… (studerar också ibland) och bor i praktiken i Gumtäkt. I Spektrum är jag i år studiesekreterare (m.m.). På hösten ska jag inleda mina pedagogikstudier och kommer kanske synas lite mindre på dagarna i Gumtäkt, men desto aktivare kommer jag hänga med på kvällarna. Lättast brukar man normalt hitta mig i kafferummet, fast jag brukar inte så ofta spela Bang!.

De tre bästa sakerna under mitt gulisår var:
\begin{enumerate}
\item Nya vänner och kontakter. Det är super roligt att skapa sig ett stort nätverk av många olika typers människor.
\item Mängder av akademiska fester man "måste" delta i, för att se till att inte missa nåt.
\item Att äntligen få ta ansvar över sina egna studier. Det var skönt att inte längre ha någon som vaktar en och kollar om läxorna är gjorda eller inte.
\end{enumerate}

\end{artikel}




\begin{artikel}{Gulisens guide till Uni}{Susanna}

\textbf{När man inleder sina studier på Uni blir man ofta bombarderad med info från höger och vänster. För att göra saker lite enklare kommer här en lista på några viktiga saker att komma ihåg.}

\textbf{Obligatoriska språkstudier}

Till kandidatexamen är det obligatoriskt att utföra 4 sp i ett främmande språk (engelska, tyska, franska, spanska, italienska eller ryska) och 3 sp i det andra inhemska språket (finska eller svenska). CEFRutgångsnivån skall vara B2 för engelska och B1 för alla andra språk. Språkstudierna kan uträttas som tent eller som kurs. 
Om man lyckades få någorlunda okej vitsord i de språk man skrev i studentexamen så lönar det sig att försöka tenta bort språkstudierna. Får man inte godkänt i tenten så kan man alltid gå kursen. Det kan även vara bra att gå tenterna redan som gulis, då gymnasiets språkstudier ännu är färska i minnet.

Tenten i det främmande språket består av en skriftlig och en muntlig del som är värda 2 sp var. Får man godkänt i ena delen men inte den andra så räcker det att gå en 2 sp kurs, annars får man gå en 4 sp kurs. Man tentar de olika delarna vid skilda tillfällen och den muntliga delen består av en hörförståelse samt diskussion. Man kallas endast till diskussion om man fick godkänt i hörförståelsen. Kurser och tenter i engelska
ordnas i Gumtäkt, och de är möjligen institutionsspecifika. Detta meddelas ofta i kurs- eller tentbeskrivningen.

Det andra inhemska språket är endera finska eller svenska. Råkar det sig så att ditt
modersmål är finska men du skrev studenten på svenska eller, vad gäller IB studerande, om du gick högstadiet på svenska så rekommenderar jag att du dubbelkollar t.ex. vid din institutions kansli vilket språk du borde studera, för chanserna är stora att du borde studera finska som andra inhemska språket fastän det är ditt modersmål. Orsaken varför det kan bli lite rörigt vad gäller språkstudier, är att det i instruktionerna ofta står modersmål när dom egentligen menar skolspråk. Så kallade modersmålsstudier, som möjligen inte är i ditt officiella modersmål, utförs i samband med kandidatexamen. 
Finska-tenten består oftast av en lista med naturvetenskapliga ord som ska översättas, samt en essä. Får man godkänt i detta så kallas man till en kort diskussion. Tenten är alltså inte så speciellt omfattande och om ens finska är någorlunda bra så lönar det sig att försöka tenta den. Som tvåspråkig upplevde jag själv tenten som väldigt enkel, engelskan var definitivt krångligare!

\newpage

\textbf{Individuell studieplan}

Den individuella studieplanen, dvs. ISP (eller HOPS som det heter på finska), är en
studieplan för kandidatexamen som ska göras upp (helst) under första studieåret. För magisterskedet görs en skild studieplan när det är aktuellt.

Man gör upp en plan med vilka kurser man planerar att gå och när man planerar att
gå dem, och efteråt godkänner ens ISP-handledare planen. Detta är som sagt en plan,
och väldigt få följer planen till punkt och pricka, så det är inget man behöver ta stress över. Få den undanstökad i början av studierna så att den inte lämnar och spöka i slutet av kandidatstudierna! På vissa institutioner fixar man ISP:n själv och på vissa ordnas någon form av handledning. T.ex. på mattan fixas den i samband med handledartutoreringen. Närmare anvisningar borde finnas på institutionernas egna hemsidor.

\textbf{IKT-körkort}

IKT-körkortet (TVT på finska) är ett bevis på att man klarar av att använda universitetets datasystem, såsom att skriva textdokument, använda bibliotekets nättjänster osv. Körkortet uträttas genom tent och är värt 3 sp. Allt material man behöver kunna finns att hitta på nätet.

IKT-körkortet kan vara ett bra sätt att bekanta sig med allt vad universitetets datasystem har att erbjuda, och studierna i detta kräver inte mycket tid. Det kan räcka med en eftermiddag eller två, beroende lite på ens utgångskunskaper och hur noggrant man läser igenom materialet.

\textbf{Stöd och boende}

Som studerande har du förstås rätt till studiestöd, som består av studiepenning, bostadstillägg och statsgaranti för studielån. Skulle det råka sig att du av någon orsak inte har rätt till studiestödets bostadstillägg så kan du möjligen ansöka om allmänt bostadsstöd. Mer information om detta kan du hitta på FPA:s hemsidor. Kom även ihåg att hålla ett öga på dina inkomster! Det är alltid lättare att låta bli att lyfta någon enstaka stödmånad än att vara tvungen att betala tillbaka eftersom du förtjänat för mycket! Studiestödet kräver också att man uträttar studier på heltid, dvs. man måste få ihop tillräckligt med studiepoäng för att kunna lyfta studiestöd.

Studerandebostäder kan hyras bl.a. via HOAS och SSBS (Svenska Studenters Bostadsstiftelse). Många nationer har även hyreslägenheter tillgängliga för sina medlemmar, så det kan löna sig att även bekanta sig med detta alternativ.

\newpage

\textbf{Biämnesstudier}

Biämnesstudier kan redan påbörjas under första studieåret, men det lönar sig att komma ihåg att vissa ämnen utanför vår fakultet är begränsade, t.ex. psykologi studier kräver inträdesprov. Information om möjliga begränsningar hittas ofta på institutionernas egna hemsidor. Ämnen från vår fakultet kan studeras fritt.
 
Vill du studera kemi så kom ihåg att det är obligatoriskt att gå Laboratoriets arbetsskydd om du vill labba. Om du vill studera data som biämne så behöver du egna lösenord till datalogens system, information för hur biämnesstuderande ska gå tillväga kan hittas på deras hemsidor under Käyttöluvat i Tietotekniikka-fliken.

\picture{images/bibba.jpg}

\textbf{Kampusbiblioteken och kursböcker}

Det finns kampusbibliotek i Centrum, Mejlans, Vik och Gumtäkt. Det som vi naturvetare har mest användning av är (förstås) Gumtäkts kampusbibliotek som finns i Physicum. Studiekortet kan aktiveras som bibliotekskort vid alla kampusbibliotek.

Det som ni främst kommer att ha användning av, som första årets studeranden, är kursböckerna som finns att låna. Kursböcker kan vara mycket dyra och detta kan vara ett bra sätt att spara pengar. Men ta i beaktan att många tänker på detta sätt så det lönar sig att vara ute i god tid, annars är alla böcker redan utlånade. Ett annat alternativ kan vara att kolla med äldre spektrumiter om de har kursböcker att låna ut eller sälja billigt. 
Gumtäkts kampusbibliotek har även en läsesal där det finns upplagor av (de flesta) kursböcker, så i nödfall kan man använda sig av dem. 

Min egen åsikt gällande kursböcker är att om man har användning av boken i flera kurser eller om det är en bra ”grundbok” som man kan använda som referensmaterial senare i studierna, så kan det möjligen löna sig att köpa den. Databöcker lönar det sig (nästan) aldrig att köpa eftersom dom ofta är väldigt dyra, föråldras snabbt och oftast kan man hitta allt material man behöver på nätet.

\textbf{Unisport}

Tycker du om att träna rekommenderar jag definitivt att du betalar Unisports träningsavgift. Med den avgiften du betalar för ett år tränar du 2-3 månader på de flesta kommersiella gym, så man sparar massor med pengar. För summan får man besöka alla Unisports gym och ta del i de träningspass som ordnas. Man får även rabatt på kurser som ordnas och olika tjänster som erbjuds, såsom anlitning av en personlig tränare. 

Första gången man betalar träningsavgiften måste man göra det på nåt av Unisports gym. Efter detta kan man betala träningsavgiften via Unisports hemsidor, om man så vill.
 
Det är ganska populärt att träna på Unisport, så träningspassen blir snabbt fulla. Det lönar sig att endera boka i god tid eller att hålla utkik ett par timmar innan passet börjar, så man kan knycka en plats som någon avbokat. Man måste avboka ett pass minst en timme innan passet börjat, och glömmer man att avboka passet så är man tvungen att betala böter. Man kan inte anmäla sig till nya pass innan böterna är betalda. 

Det lönar sig att använda gymmet under opopulära tider (mitt på dagen, strax före stängningsdags), för efter fyra brukar gymmet vara helt fullpackat och man kan vara tvungen att köa till maskiner.

\end{artikel}





\begin{artikel}{Spektrums polisavdelnings
brottsregister}{}
\begin{twocolumns}

\columnpicture{images/lisal.jpg}

\textbf{Namn:} Lisa Lindfors

\textbf{Kända alias:} Lucky Lisa, Smaug, Skattmästare

\textbf{Födelseort och -tid:} Mariehamn, 24.06.1992

\textbf{Längd:} 177 cm

\textbf{Hårfärg:} N/A

\textbf{Utmärkande drag:} Tycker om heavymetal och hästar, allmänt härjig

\textbf{Senast kända vistelseort:} Campus Vik

\textbf{Tidigare domar:}

		2012	Mordförsök, Förgiftande av mat i värdinneriet
		
		2012 	Störande av allmän ordning, PRK	

\textbf{Kommentar:} Färgar håret och byter studieinriktning ofta, för att undgå lagens långa arm. I den sk. Föreningen håller hon hårt i slantarna och överser penningtvätten.



\columnpicture{images/fannyb.jpg}

\textbf{Namn:} Fanny Bergström

\textbf{Kända alias:} Fanny von Frankenstein, Studiesekreterare

\textbf{Födelseort och -tid:} Lovisa, 04.04.1992

\textbf{Längd:} 167 cm

\textbf{Hårfärg:} Ljus

\textbf{Utmärkande drag:} Har ofta pomppufiilis

\textbf{Tidigare domar:}

		2012	Urkundsförfalskning, Sekreterare
		
		2013	Människohandel för tvångsarbete, Ordförande
		
		2014	Spridande av copyrightskyddat material, Sångboksgeneral

\textbf{Kommentar:} Galen vetenskapskvinna. Använder sig ofta av flytande kväve eller torris i sina sk demonstrationer.



\newpage



\columnpicture{images/johanb.jpg}

\textbf{Namn:} Johan Blomqvist

\textbf{Kända alias:} Johan the Kid, Turncloak John, Programchef

\textbf{Födelseort och -tid:} Borgå, 19.03.1993

\textbf{Hårfärg:} Mörk

\textbf{Längd:} 180 cm

\textbf{Utmärkande drag:} Mössa med tofs, vinröda halare

\textbf{Tidigare domar:}

		2013 Spritlangning
		
		2013 Störande av allmän ordning, PRK
		
		2013 Hets mot folkgrupp, Sångledare
		
		2014 Sexuellt trakasseri, Jämställdhetsansvarig
		
		2014 Förledande av ungdom, Tutor
		
		
\textbf{Senast kända vistelseort:} Otnäs

\textbf{Kommentar:} Ett brottsligt allt-i-allo.


\columnpicture{images/jonasw.jpg}

\textbf{Namn:} Jonas Westerlund

\textbf{Kända alias:} Tequila Jonas, Jonas Delgado, Ordförande

\textbf{Födelseort och -tid:} Okänd, 06.07.1991

\textbf{Längd:} 5,8 fot

\textbf{Hårfärg:} Salt and Pepper

\textbf{Utmärkande drag:} Final Fantasy -expert

\textbf{Tidigare domar:}

		2012	Frihetsberövande, Studiesekreterare
		
		2012	Smuggling, Teleportansansvarig
		
		2014	Människohandel för tvångsarbete, Ordförande

\textbf{Kommentar:} Föreningens Al Capone. Är varken rädd för Pünschen eller Kolven.






\columnpicture{images/fransg.jpg}

\textbf{Namn:} Frans Graeffe

\textbf{Kända alias:} Bullseye Graeffe, Caps the Killer, Jämställdhetsansvarig

\textbf{Födelseort och -tid:} Någonstans nära Aura Å, ca sista september 1994

\textbf{Längd:} 64,9606299 tum

\textbf{Hårfärg:} Ljus

\textbf{Senast kända vistelseort:} Capsringen

\textbf{Tidigare domar:} inga

\textbf{Efterlyst sedan:} 2014

\textbf{Kommentar:} Sköter föreningens smutsjobb. En träffsäker, känslokall mördare som inte skonar barn eller kvinnor.



\columnpicture{images/mirjamk.jpg}

\textbf{Namn:} Mirjam Kauppila

\textbf{Kända alias:} Ten-finger Mirkku, Sekreterare

\textbf{Födelseort och -tid:} Borgå, Maj 1996

\textbf{Längd:} 164,77 cm

\textbf{Hårfärg:} Mörk

\textbf{Utmärkande drag:} Har en förkärlek för vodka-godisnallar

\textbf{Tidigare domar:} inga

\textbf{Efterlyst sedan:} 2014

\textbf{Senast kända vistelseort:} Vallgård

\textbf{Kommentar:} Misstänks förfalska dokument för föreningen.

\end{twocolumns}
\end{artikel}


\begin{artikel}{Pseudointellektuellt svammel}{Jere}

\textbf{År 1 735 publicerade den svenske vetenskapsmannen Carl Linnaeus (observera att han adlades först år 1 757 [1]) den första upplagan av kategoriseringsverket Systema
Naturae. Han var den första att gruppera människorna och aporna i samma släkte.} Intressant nog skiljer han inte åt oss från våra kusiner hankeiterna utgående från biologiska olikheter, utan med filosofiska aforismen nosce te ipsum, känn dig själv. Han menar att självkännedom är det definierande draget för oss som en art[2].

Aforismen ifråga är dock äldre än de knappa trehundra år emellan oss och upplysningens Sverige. Det är uppenbart att aforismen var viktig även för antikens greker, då begreppet var inskrivet i väggen vid Apollons tempel i Delfi[3].

Begreppet ”känn dig själv” förekommer i många variationer, t.ex. i filmen The Matrix (1999).

Den skarpa läsaren frågar sig varför detta är relevant för ett gulisnummer. Studenttidningar och dylika infoblad brukar innehålla en hel del goda råd åt de nya. Skaffa bostad, kom ihåg att delta i fritidsverksamheten, gnäll om pengar av FPA osv. Bland allt detta kan det vara svårt att komma med något revolutionerande och intressant. En svår målgrupp för skribenter, vilket i första hand beror på att de naturvetenskapliga gulisarna tenderar att vara oskulda. Alltså vad gäller studielivet. Det är krävande att förklara hur allt fungerar, lite som att förklara åt ordningsvakten att nej, det var varken du eller din kompis som spydde i hörnsoffan. Kemister destillerar olika saker med varierande framgång och här följer mitt försök att destillera en gnutta sanning.

Sun Zi, en kinesisk general född ca 500 f.Kr., skrev klassikern Krigskonsten. I detta epos konstaterar han att ifall man känner sig själv och fienden, behöver man inte frukta resultatet av hundra strider[4]. Men vem är fienden? Det kan vara FPA med deras inkomstgränser från helvetets åttonde krets eller kanske ens egen lathet inför tenten. Faktum är att motståndarna är många och av varierande form, men självkännedom om ens styrkor och svagheter är konstant.

Och här kommer vi tillbaka till att känna sig själv, då det kan vara svårt att veta vad man vill. Studielivet är dock en utmärkt chans att vidga sina vyer. Många lär sig sina fysiska gränser på festernas sena timmar. Andra lär sig mentala dygder, såsom tålamod, genom sin första styrelsepost (det bör påpekas att det som händer i styrelsen stannar i styrelsen). Detta för att nämna några exempel.

Risken finns att självkännedom är ett fenomen av zenbuddistisk kaliber. Man kan inte bli upplyst om man ämnar bli upplyst, utan man måste snubbla på sanningen av misstag. Inte så långt ifrån att snubbla på tamburmattan i morgonmörkret.

Men hur borde man gå till väga? Det finns knappast ett definitivt svar. Mitt råd lyder att pröva olika saker, även om man tvekar, t.ex. tidigare nämnda styrelseposten. Av erfarenhet är det känt att ofta blir man positivt överraskad. Olika upplevelser avslöjar nya sidor av ens karaktär. Besök även andra föreningar, där man med god sannolikhet lär känna färggranna personer. Ofta har de annan inställning än en själv till flera frågor, vilket i bästa fall leder till ögonöppnande aha! -upplevelser. Även Paul Bragiel, VD för i/o Ventures, ett av de största investeringsbolagen i Silicon Valley, konstaterade att det bästa med universitet är att träffa spännande personer.

Spektrum är en liten men hårt sammansvetsad förening, vilket gör det enkelt att lära känna de andra. I mitt tycke är det också ett utmärkt sätt att utmana sig själv och knyta nya vänskapsband. Alla är välkomna att delta i verksamheten, även om de inte ämnar fortsätta studera naturvetenskaper. Några av er kommer att hitta er plats i livet på något annat ställe. För er vill jag citera Marcus Aurelius (1 21 – 1 80 e.Kr.), romersk kejsare och en av de få män som uppfyller Platons utopistiska vision om en filosof som kung[5]: 

Även om du gett upp hoppet om att bli en stor tänkare eller vetenskapsman, ge inte upp hoppet om att uppnå frihet (7.67)

Referenser
\begin{enumerate}
\item Blunt, W. Linnaeus: The Compleat Naturalist, Frances Lincoln Ltd., 2004, s. 1 71

\item Klein, R.A. Sociality as the Human Condition: Anthropology in Economic, Philosophical and Theological Perspective, Koninklijke Brill NV, 2011 , s. 59

\item Miller, J. Examined Lives: From Socrates to Nietzsche, 1. t., Farrar, Straus and Giroux 2011, s. 22

\item Sawyer, R.D. The Seven Military Classics of Ancient China, Basic Books, 2007, s. 421 – 422

\item Aurelius, M. Meditations: A New Translation, övers. Hays, G., Modern Library, 2002, s. i
\end{enumerate}
\end{artikel}


\begin{artikel}

\picture{images/phd.jpg}

\end{artikel}




\begin{artikel}{Medlemspresentation}{}
\begin{twocolumns}

\textbf{Eftersom det är mycket nytt när man är gulis, så tänkte vi att det kunde vara bra med en presentation av aktiva medlemmar som på ett eller annat sätt utmärker sig. Detta är alltså inte alla Spektrums medlemmar! Vi bad helt enkelt Spektrums medlemmar att skicka in en kort textsnutt om andra medlemmar och här är nu en sammanställning av dem. Förhoppningsvis gör presentationerna det lättare att lära känna och känna igen spektrumiterna. Om inte, så känns de igen på att de i princip är de enda som talar svenska uppe i Gumtäkt...}

\textbf{Aino} – Matematikstuderande som gillar dammiddagar och kanelbullar från Physicum

\textbf{Andreas Forsblom} - Datalog som cyklar och dricker vin, men inte samtidigt. Är sällan långt ifrån en dator eller en kamera. RGL

\textbf{Annika Venäläinen} - Fysiker som skulle bli lärare. Blev istället fast i acceleratorträsket. RGL

\textbf{Ansku} – En matematiker som spelar rollerderby och som har bevisat att man inte behöver vara finlandssvensk för att vara spektrumit.

\textbf{Axel} – Fick smeknamnet Pietari under analyshandledningen och verkar vara hemma med matematik.

\textbf{Camilla (Backman)} – Vår partypinglande klubbmästare från Pampas som extraknäcker här och var.

\textbf{Celina} – Fysiker som brukar synas till här som var, flitig prk-medlem och även chefredaktör.

\textbf{Cessi} – jobbar på Dynamicum med att bestämma/förutspå vädret. Dyker också hon upp vid idrottstillfällen.

\textbf{Christian ”Chride” Saxén} - Matematikergamyl som uppskattar goda drycker och fysiska tangentbord. RGL

\textbf{Emppu} – Spektrumit mer eller mindre mot sin vilja. Verkar dock trivas bra bland ”nördarna” (som hon själv kallar spektrumiterna)

\textbf{Fanny (Antell)} – Betare, som blivit spektrumit. Hittas bäst i Vik eller Arabiastranden.

\textbf{Fanny} - Kemist som anväder en stor del av dygnet till jobb i Gumtäkt som inte har med studierna att göra. Lyckas ändå på något sätt hitta tid för att delta i olika händelser, sköta om små och stora gulisar och till och med tappa bort sig i Jukolaskogen.

\textbf{Filip} - Fysiker som är tronföljaren i den mytomspunna Granberg-ätten.

\textbf{Frans} - 

\textbf{Fredi} – ”Granberg den äldre”, f.d. ordförande och aktiv fotograf. Försöker göra vetenskap i acceleratorlabbet. RGL

\textbf{Henrika} - Kemist som gjorde vågor genom att emigrera till Aalto. RGL

\textbf{Iris} - Ex. matematiker som nuförtiden studerar på statsvetenskapliga fakulteten. Kan ändå ofta hittas i kafferummet, på olika fester eller på klätterväggen.

\textbf{Jenny} - Denna häcklöpande fysiker från västra Nyland kan enklast hittas i kafferummet, på klätterväggen eller ibland t.o.m. i Jukolaskogen

\textbf{Jere} – Kemist som kan få vem som helst att rodna djupare än Marianergraven.

\textbf{Jeremias} – Lång matematiker som leker datalog emellanåt. Tidigare har han varit bl.a. ordförande och redaktionschef.

\textbf{Joel} – var matematiker en gång i tiden, men insåg att det inte var hans grej. Syns till vid idrottstillfällen, särskilt om det är basket på agendan.

\textbf{Johan} – Programchef som fick solsting och började studera på Aalto. Tycker om gamla spel.

\textbf{Jonas} – Ordförande som kan teleportera, har nu gjort det två gånger.

\textbf{Kadri} – Medlem i redaktionen som skriver väldigt intressanta artiklar. Talar gärna svenska.

\textbf{Kasper} – En snäll fysiker från Ingå, som spelar kanotpolo och lagar god mat på sitser.

\textbf{Kati} – Orförandets syster och självutnämnd beerpong mästare. Ses ibland på Klubben och då oftast i full gång med beerpong.

\textbf{Kimmo Kettula} - Doktorand i astronomi som tidvis trollar fram en GT (med betoning på G). RGL

\textbf{Klaus} - 

\textbf{Kristian ”KM” Meinander} – Sportglad kemidoktor som allt emellanåt syns på evenemang. F.d. ordförande, programchef och klubbmästare. RGL

\textbf{Lasse} - Matematiker så positiv att han ej ser problem, endast inversproblem. RGL

\textbf{Laura (Granberg, f.d. Lindfors)} – Färdigutbildad läkare, men syns till ibland vid Fredis sida.

\textbf{Laura} – Glad hästtjej som ses hänga i kafferummet.

\textbf{Lisa} - Åländsk ex. matematiker som nuförtiden läser farmasi. I år tar hon hand om spektrums pengar, något som hon hittills skött mycket väl. Vill man hitta henne  är det bäst att bege sig till centrum, antingen till Klubben eller Nationerna.

\textbf{Mikka} – Chilipili, ingen mer förklaring behövs. (xqrrfürer 2013) RGL

\textbf{Mindi} – Har lovat ordna herrmiddag i en svag stund i sitt liv, ses annars dela bulle med Aino vid Physicum

\textbf{Mirjam} "\textbf{Mirkku"} - 

\textbf{Niklas} - Cyklande datalog med långa biämnen i datatutori och ADB-ansvarighet.

\textbf{Oskar (Kortelainen)} – Matematiker som vill bli lärare. Syns väldigt ofta till i kafferummet pratades med allt och alla.

\textbf{Paul} – Matematiker som detta år kommer att fungera som tutor. Hittas ofta i kafferummet spelandes BANG!

\textbf{Robert} – fysiker som under dammiddagen gjorde succé som ”Chéf Robért”. Syns ofta till i Physicums café och i Unisport.

\textbf{Roope} – Matematiker som aldrig syns till, förutom på Ylonz där han blir av med alla sina pengar. Nu även kafferumsansvarig.


\textbf{Sandra} - En alltid glad och pigg pampes som gillar god bål, gott sällskap och Final Fantasy. Den ena halvan av den före detta Vik-maffian från Pampas.

\textbf{Sebbe (Björkqvist)} – Sebbe är gurun som alla söker sig till efter råd och stöd när sådant behövs.

\textbf{Sebbe (Falk)} – Matematiker från Hangö som brukar ses träna på gymmet mer ofta än sällan.

\textbf{Stefan Michelsson} – Matematikergamyl som både sjunger och skriver snapssånger. RGL

\textbf{Stefan Sjölund} - Spektrums största och hårigaste ålänning. Gillar öl, whisky och metal.

\textbf{Sonja} – En supersportig spektrumit som gärna hjälper till i köket under sitser. Debuterade i Spektrums Venla-lag detta år.

\textbf{Tobias} -

\textbf{Toffe} - hör till gruppen "glada människor från Pampas". Hittas i ellabbet, i Physicums café eller annars bara i Gumtäkt. RGL





\end{twocolumns}
\end{artikel}




\begin{artikel}{En finskspråkig flicka i en svenskspråkig förening}{Ansku}
\begin{twocolumns}

De 19 första åren av mitt liv bodde jag i en liten by i ett helt finskspråkigt område i Satakunta. Där finns det inga finlandssvenskar och inga invandrare. Även människor som talar finska med någon tydlig dialekt är svåra att hitta. Ni kan troligen gissa att den första gången jag hörde någon prata svenska var via TV:n. När jag var liten var BUU-klubben roligt att titta på då och då, även om jag inte förstod någonting.
 
Den första gången jag kommer ihåg att jag verkligen hörde någon prata svenska var när jag var tolv år. Det var sommaren före högstadiet. Jag var redan rädd för den nya skolan och för att jag måste börja studera svenska och andra nya ämnen. Jag försökte förbereda mig så bra som möjligt. Till exempel åkte jag till Kristinestad med min mamma och pappa. Det var ett lyckat språkbad.

Hösten kom och jag började studera svenska med mina nya klasskamrater. Tiden gick snabbt och efter sex långa år fick jag min studentexamen. Under åren hade jag lärt mig att lärare inte alltid är så bra på sina jobb och hur många elever faktiskt hatar svenska. Vi diskuterade sällan under lektioner och bara en gång hörde jag svenska utanför klassrummet, det var under vår klassresa till Stockholm i slutet av högstadiet. Således är det inte överraskande att jag inte var så säker med mina språkkunskaper.

\columnpicture{images/ansku.jpg}[4cm]

Nu var det dags att flytta till Helsingfors och börja studera matematik. Under mitt första år träffade jag några svenskspråkiga studenter. En sak ledde till en annan och jag blev medlem i en förening som kallas Spektrum. Jag var ännu för blyg att egentligen använda svenska men jag fick nya vänner trots det och hade det så roligt.

Nu kan man fråga sig att varför jag vill hänga med de svenskspråkiga? För det första vill jag tillbringa tid med dessa vänliga människor, för det andra så är det roligt att se hurudana vanor andra föreningar har. Tack vare Spektrum har jag lärt mig att inte alla sits-sånger måste vara långa och fulla av verser och ordet ”nyt”, att caps är kul, att man kan sitsa utan Jaloviina, att små föreningar är lika bra och viktiga som de stora och oräkneligt många andra saker.

Jag kan föreställa mig att många finskspråkiga studenter har en liknande historia med det andra inhemska språket som jag. Vi känner oss osäkra i situationer där svenska förekommer, men det betyder inte att vi borde vara rädd för dessa situationer. Ett besök i en annan förening kan ge dig nya värdefulla vänner och fantastiska upplevelser, till exempel en möjlighet att skriva en artikel för en studenttidning.

\end{twocolumns}

\picture{images/rubber_sheet.png}[][xkcd.com]

\end{artikel}





\begin{artikel}{Presentation av ämnesföreningar}{}

Som kanske inte så många av er vet så finns det fler ämnesföreningar inom Gumtäkts
väggar. Spektrum är den enda helt svenskspråkiga, men finskspråkiga finns fler än väntat. Så
om man vill finslipa det andra inhemska språket lite så talar de finskspråkiga studerandena
mer än gärna med en. Här har en del av de finskspråkiga förningarna skrivit en liten
presentation om deras förening, på svenska! Ta er en titt och kanske finner ni en ny
vänförening.

\textbf{Limes}

Limes är en ämnesförening från Gumtäkt som grundades redan 1 936.
Om du studerar vad som helst i Gumtäkt är Limes din förening. ;)
Vi organiserar olika evenemang - från bastukvällar och exkursioner till sitzer och bileen.
För oss är det största evenemanget Limeksen Appro med sin 2000-3000 årliga deltagare.
Vi säljer också läroböcker och halarmärken förmångligt.
Välkommen att bli medlem vid vårt kontor som ligger i Exactum, rummet C132! <3

\textbf{HYK}

Helsingin Ylipiston Kemistit ry eller HYK är en finskpråkig ämnesförening för kemister i
Gumtäckt.
HYK är grundad år 1 927 och är en av de äldsta studentförenringar i Helsingfors
Universitet.
Vi älskar traditioner men också att oganisera nya händelser förl våra medlemmar.
Man hittar kemister i svarta overaller i Opsos, HYKs kafferum, som ligger i Chemicum.
Välkommen!

\textbf{Geysir rf}

Är du intresserad av jorden och dess forskning? Geofysik är en vetenskap som forskar kring jordens fysik. Hos oss kan du exempelvis studera de fysikaliska egenskaper hos sjöar, älvar, hav, oceaner, snö och is, samt jordskalv,jordklotets form, polarsken och jordens magnetfält. Därtill finns det i geofysik en rad viktiga tillämpningar för de som är fascinerade av programmering och matematiska modeller. Geofysiker får röra sig även utanför campusområdet, då fältkurser arrangeras årligen både i Finland och utomlands. Mot slutet av studierna blir geofysiker erbjudna examensarbetsmöjligheter på många exotiska platser, såsom Antarktis eller Spetsbergen.

Geysir rf grundades år 1 997 och är blivande geofysikers intresseorganisation. Vårt mål är
att ordna så mångsidig underhållning som möjligt för våra medlemmar: exkursioner, dvs
studiebesök hos studie- och forskningsanstalter samt på isbrytare och på havsforskningsfarty, idrott (bl. a. väggklättring), kultur (filmer, teater, konserter) samt soaréer (film- och spelkvällar). Exkursionerna till utlandet ingår sedan länge i Geysirs traditioner. Sådana besök inträffar vanligtvis med två års mellanrum. Vi har nyss varit på Island, vid Spetsbergen, på Azorerna, på Nya Zeeland och i Etiopien. Målet för resan år 2014 var Ungern, och var och en är välkommen att komma med idéer till vår nästa resa för år 2016!

\textbf{Synop ry}

Synop ry är meteorologiska ämnesorganisationen som har hjälpt människor rikta sina
ögon mot moln för redan 45 år. Vi är ganska liten, men aktiv organisation som ordnar olika
slags evenemang från sport och filmkvällar till bingor. Den största bingon ordnas varje vapp
och alla är välkomna. Mer information om oss kan ni hitta från vår webbsida www.atm.helsinki.fi/synop eller studentrumet i Physicum där det ofta finns några medlemmar.

\textbf{Meridian rf}
Tycker du om att vara vaken helan natten och att sova hela dagen? Meridian rf
(Meridiaani) är en ämnesförening för astronomistudenter i Helsingfors universitet. På hösten
har vi olika slags evenemang för unga astronomer, vi grillar, spelar brädspel och vår ölklubb träffas varje månad. På våren organiserar vi ett berömd rymdparti Yuri's Night. För att observera stjärnor, planeter och galaxer användar vi en 60-cm teleskop i Metsähovi
Obsevatoriet i Kyrkslätt. Du kan hitta oss i Kumpula i Physicum's Studentrummet OH med de
andra fysikerorganisationerna.
\end{artikel}



\begin{artikel}{Byggnader i Gumtäkt}{Sandra}

Hej alla nya studeranden och även ni lite äldre. Här kommer en liten presentation över
byggnaderna som finns vid Gumtäkts campus. Så om ni aldrig har varit där tidigare,
eller inte tillbringat tid där på så länge att ni inte minns hur det ser ut mera, så kan
det vara en bra idé att ta er en titt på denna artikel.

\textbf{Exactum}

Det är här alla matematikrumpnissar och datalogvildvittror håller hus. Djävulsbacken upp
till denna byggnad brukar vara lika trevlig som ett myggbett att komma upp för, speciellt om
man är sen och inte gillar att röra på sig. Andra våningen är mest datalogernas territorium,
medan matematikerna rör sig på tredje våningen. ”Pajan”, som också finns på tredjevåningen,
är matematikernas räknestuga. Där kan man sitta och studera om man har tid för sådant och
även få hjälp av assistenter på plats. På första våningen finns det två stora föreläsningssalar där också de flesta tenterna skrivs. Matsal finns även här, om än en lite mindre sådan. Exactum är överlag mest likt en gaffel, men ett helt trevligt ställe om man gillar sådana. Dessutom finns Spektrums kafferum på första våningen. Där kan man alltid sitta och umgås med likasinnade svenskpråkiga studeranden.


\picture{images/physicum.jpg}

\textbf{Physicum}

Här styr och ställer grådvärgsfysikerna. Ska man till Physicum brukar oftast den högra
delen av djävulsbacken vara den rätta vägen att ta. I Physicum finns det både det ena och det andra. Biblioteket, där man förvånansvärt nog kan låna böcker eller bara sitta och studera ostört, samt ett café, där det säljs paninin, baguetter (eller "patonkin" som de i folkmun också kallas) och goda bullar, finns här. ”Sandlådan” är också ett ställe där flitiga studeranden kan samlas; denna del finns på högsta våningen i Physicum. Det finns självfallet också utrymme för klassrum. I dessa klassrum finns både fysiker, geologer och andra fysikintresserade människor.

\picture{images/chemicum.jpg}[8cm]

\textbf{Chemicum}

Kemiströvarnas härbärge. Byggnaden är åtskild från de övriga, vilket brukar resultera i att
endast kemister rör sig i ”Chemen”. Om vi andra studeranden drar oss ditåt är det för det
mesta för att stilla hungern i matsalen. Svenska chemen fanns tidigare i denna byggnad, men
tyvärr har vi fått säga adjö till detta svenskspråkiga fenomen. Spektrums kafferum fanns
tidigare Chemicum.

Andra byggnader som man kan tänkas stöta på när man rör sig vid Gumtäkt är bl.a.Unisport, acceleratorlaboratoriet och kansliet. Unisport är stället dit alla sportiga och ibland
icke-sportiga rör sig för att röra på fläsket. Kansliet är dit ni borde gå för att bland annat få ett fint klistermärke på ert studiekort. Om man har andra frågor är det också dit man borde röra sig. Acceleratorlabbet är stället alla viktiga fysiker samlas på. Vem de är och vad de gör där vet ingen, men rykten går kring Gumtäkt som en löpeld.


\end{artikel}

\newpage

.








%----------------------------------------------------------------------
%	This command prints the publisher information on one line with
%	all fields separated by 'bullets'.
%----------------------------------------------------------------------
\lastpage{0.9\textwidth}{images/Lalreklam.pdf}

\end{document}